
\section{メタアナリシスとは}
\label{sec:2}
メタアナリシスは複数の臨床試験の結果を統合する統計手法の事である.
臨床試験の結果としては,薬剤や治療法の治療効果や有害事象の出やすさなどが想定され,以下ではこのことを治療効果と書くことにする.

メタアナリシスの利点はいくつか考えられる.
統合によって高い精度で治療効果の推定を行うことができる点は有益であるといえる.
また,異質性の情報を正しく考察する事で,対象全体に同じような効果が期待できるのか,異なる治療効果を持ったサブグループが存在するかなどの臨床上有用な情報を得られる点も有益であるといえる.
これらの情報は個々の試験を見ているだけでは得られない情報である.
複数の試験結果が存在する場合,その時点でわかっていることをレビューとしてまとめたうえで,統合によって新たに分かることがないかを調べることは必須であると考えられる.

メタアナリシスには当然欠点もある.
統合できないような異質な試験を集め,バイアスのある結果を得てしまうことは問題である.
ネガティブな試験の結果は公表されにくく,これに起因する公表バイアスの問題は治療効果を過大評価する危険性がある.
これらの問題が起きないような対処が重要である.
そのためには,統合のための研究を実施する前にプロトコル(計画書)を作成し,検索対象や検索式を事前に規定したり,文献選択ルールを取り決めたり,選択された文献のRisk of Bias評価の方法であったり,未公表のデータをいかにして入手するかを考えたりする必要がある.
このような質の高いシステマティックレビューを実施した後に,可能な場合に限りメタアナリシスを実施することが重要である.
Cochrane Hand Book \citep{Higgins2019}のメタアナリシスの節の最初,Section 10.1のタイトルは``Do not start here!''である.

現在では,システマティックレビューとメタアナリシスの資料は充実してきている.
オンライン版のCochrane Hand Book \citep{Higgins2019}は無料で閲覧でき,研究の事前登録サイト\citep{PRISMA}や報告のためのガイドライン\citep{PROSPERO}も整備されてきた.
また,ソフトウェアも充実してきており,一般的にはCochrane Collaborationが提供しているRevMan \citep{RevMan}やRやStataを用いて解析が行われている.
また,Rについては最近フリーの電子書籍が出版されている\citep{Harrer2019}.
ただし,本稿のスコープは数理的な解説であるため,実務の話はこの程度にとどめておく.
「質の高い研究に必須かつ研究の根幹を成す」実務の解説については,別の情報を参照していただければ幸いである.
